% -------------------------------------------------------
%  Abstract
% -------------------------------------------------------

\vspace*{-60pt}
\enlargethispage{90pt}

\شروع{وسط‌چین}
\مهم{چکیده}

\پایان{وسط‌چین}
\vspace{-1cm}

\بدون‌تورفتگی

در مقایسه با رویکردهای پیچیده‌ای که معمولاً برای انتقال دانش از شبکهٔ معلم به شبکهٔ دانش‌آموز به کار می‌روند، این مقاله روش ساده و در عین حال قدرتمندی را برای بهره‌گیری از نقشه‌های ویژگی پالایش‌شده جهت انتقال توجه (Attention) معرفی می‌کند. روش پیشنهادی، موسوم به استخراج ویژگی با راهنمایی توجه Distillation) Feature Attention-guided یا (AttnFD، از ماژول توجه بلوک‌های کانولوشنی (CBAM) که ضمن پالایش نقشه‌های ویژگی، هم اطلاعات کانالی و هم فضایی را در نظر می‌گیرد، بهره می‌برد. علاوه بر این، برای افزایش ظرفیت مدل در استخراج الگوهای مکانی و فرکانسی، از رویکرد شکافت طیفی عمیق Decomposition) Feature Spectral (Deep نیز استفاده می‌شود. در این چارچوب، تنها با به‌کارگیری تابع هزینهٔ میانگین مربعات خطا (MSE) بین نقشه‌های ویژگی پالایش‌شدهٔ معلم و دانش‌آموز، ترکیب روش AttnFD و مدل طیفی عمیق توانسته است در وظیفهٔ تقسیم‌بندی معنایی Segmentation) (Semantic عملکرد چشمگیری نشان دهد و با ارتقای میانگین تقاطع بر اتحاد (mIoU) در شبکهٔ دانشجو، نتایجی هم‌تراز یا حتی بهتر از روش‌های مطرح کنونی به دست آورد. آزمایش‌ها روی مجموعه‌داده‌های متنوع نظیر PascalVoc 2012، Cityscapes، COCO و CamVid بیانگر برتری این رویکرد و دستیابی به نتایجی پیشرو در زمینهٔ بهبود دقت تقسیم‌بندی معنایی هستند.

\پرش‌بلند
\بدون‌تورفتگی \مهم{کلیدواژه‌ها}: 
بینایی کامپیوتر، یادگیری ژرف ، تقطیر دانش، بخش‌بندی معنایی 
\صفحه‌جدید

% چکیده فارسی
\section*{چکیده}
سامانهٔ «پورتفولیو تجمیعی» یک بستر وب محور است که دارایی‌های کاربر را از طبقات مختلف—سهام، طلا، رمزارز و املاک—در قالب یک داشبورد یکپارچه نمایش می‌دهد و با بهره‌گیری از الگوریتم رشد سی‌روزه، توصیه‌های هوشمند «خرید»، «نگهداری» یا «کاهش موقعیت» ارائه می‌کند. این پژوهش روند طراحی و پیاده‌سازی سامانه را تشریح می‌کند: از معماری میکروسرویس مبتنی بر Docker Compose تا لایهٔ توصیه‌گر در سرویس API. نتایج ارزیابی‌ تجربی نشان می‌دهد پیشنهادهای سامانه با شاخص سودآوری سالانهٔ ۱۲٪ در سبد نمونه، عملکرد بهتری نسبت به استراتژی خرید و نگهداری داشته است. این سیستم می‌تواند به‌عنوان ابزاری کارآمد برای سرمایه‌گذاران خرد در بازار ایران به‌کار رود.
